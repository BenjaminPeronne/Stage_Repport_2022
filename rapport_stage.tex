\documentclass[a4paper, 11pt]{report}

\usepackage[T1]{fontenc} % Times New Roman
% \usepackage{times} % Times New Roman
\usepackage{lmodern} % sans serif
\usepackage[utf8]{inputenc} % UTF-8
\usepackage[french]{babel} % French 

\usepackage{minitoc} % Table of contents
\usepackage{geometry} % Page size
\usepackage{ragged2e} % Ragged right
\usepackage{fancyhdr} % Header
\usepackage{graphicx} % Images
\usepackage[scaled]{helvet} 
	
\usepackage{listings} % Code


\graphicspath{ {./images/} }

% \UseRawInputEncoding

\justifying
\setlength{\headheight}{0.5cm}
\setlength{\headsep}{0.5cm}

\geometry{hmargin=2.5cm,vmargin=2.5cm}

\setcounter{secnumdepth}{3} % Section number depth
\setcounter{tocdepth}{1} % Table of contents depth
\setcounter{minitocdepth}{3} % Table of contents depth

\begin{document}

% Intro             ===============================
\begin{titlepage}
    \begin{center}
        \vspace*{1cm}

        \textbf{\Huge{ISYEB}}

        \vspace{0.5cm}

        \textbf{Rapport de stage de 3ème année de Licence Informatique} 

        \vspace{0.5cm}

        \small{\textit{Sujet : Réalisation d'un dispositif de capture vidéo pour l’acquisition de données dans le cadre d’une manipulation en biologie.}}

        \vspace{1.5cm}

        \textbf{Aymerick LAURETTA-PERONNE} \\

        \vfill

        % Rapport de stage de 3ème année de Licence Informatique \\

        \vspace{0.8cm}

        \includegraphics[]{logo.png}
        
        \textbf{Informatique} \\
        \textbf{Université des Antilles} \\
        \textbf{Région de la Guadeloupe} \\
        \vspace{0.5cm}
        \textbf{\today {}}
         
    \end{center}
\end{titlepage}
% Intro             ===============================

\dominitoc
\tableofcontents



\chapter{Introduction}
\minitoc       

    \section{Présentation}

\chapter{Environnement}
\minitoc

    \section{Présentation de l'entreprise}
    \section{L'équipe projet}
    
\chapter{Conception et réalisation de l'application}
\minitoc 

    \section{Description, résultats attendus et objectifs}
    \section{Etude du besoin}
        \subsection{Contexte}
        \subsection{Analyse du besoin}
        \subsection{Définition des besoins}
        
    \section{Choix des technologies}
        \subsection{Choix du langage Python}
        \subsection{Choix l'API PiCamera}
        \subsection{Choix de la bibliothèque Tkinter}

\chapter{Présentation de la problématique (sujet du stage)}
\minitoc
    \section{Projet : Montage du dispositif de capture vidéo}
      
        \subsection{Objectifs}
            \begin{itemize}
                \item Montage du dispositif de capture vidéo
                \item Acquisition des données
                \item Traitement des données
                \item Visualisation des données
            \end{itemize}
    

    \section{Projet : Rélisation du logiciel de capture vidéo}
        \subsection{Objectifs}
            \begin{itemize}
                \item Réalisation du logiciel de capture vidéo
                \item Acquisition des données
                \item Traitement des données
                \item Visualisation des données
            \end{itemize}

\chapter{Travail réalisé}
\minitoc      

\chapter{Conclusion}
\minitoc      

\end{document}