\chapter{Présentation de la problématique (sujet du stage)}
\minitoc

    \section{Description, résultats attendus et objectifs}
    \section{Etude du besoin}
        \subsection{Contexte}
        \subsection{Analyse du besoin}
        \subsection{Définition des besoins}
        
    \section{Choix des technologies}
        \subsection{Choix du langage Python}
        \subsection{Choix l'API PiCamera}
        \subsection{Choix de la bibliothèque Tkinter}
        \section{Projet : Montage du dispositif de capture vidéo}
      
        \subsection{Objectifs}
            \begin{itemize}
                \item Montage du dispositif de capture vidéo
                \item Acquisition des données
                \item Traitement des données
                \item Visualisation des données
            \end{itemize}
    

    \section{Projet : Rélisation du logiciel de capture vidéo}
        \subsection{Objectifs}
            \begin{itemize}
                \item Réalisation du logiciel de capture vidéo
                \item Acquisition des données
                \item Traitement des données
                \item Visualisation des données
            \end{itemize}