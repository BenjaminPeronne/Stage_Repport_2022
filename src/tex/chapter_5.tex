\chapter{Conclusion}
    % (il faut dire à un moment a quoi ca sert tout ça :
    % automatiser la detection des gerridés par traitement des video -> en temps réel !!)
    Ce stage a permis le début du développement d'un système de reconnaissance de gerridés afin de faciliter le travail des biologistes durant leur expérimentations.
    
    % Finaliser section 1
    % \section{Interface labo / traitement informatique}

    \section{Ressentis par rapport au stage}
    Le travail scientifique que je réalise durant mes études mon permis de comprendre les différentes documentations allant du montage du Raspberry Pi à la programmation et l'utilisation de l'API PiCamera afin de concevoir l'application. 

    % \begin{flushleft}
        % Nos divers formation en gestion de projet, mon permis de réussir le travail demandé sous contrainte de temps.

        % \vspace{0.2cm}

        % Notre apprentisage universitaire en informatique ma permis de m'adapter rapidement ainsi que de travailler de manière efficace.

        % \vspace{0.2cm}

        % Bien que la formation universitaire permet de bien s'adapter a la vie professionnelle, ce stage en immersion ma permit de dévoiler d'autres facette de mon futur métier. 
    % \end{flushleft}


    \section{Perspectives futures}
    Durant ce stage, nous avons remplacer un dispositif permettant la capture vidéo de qualité en terme de vidéo grace à la GoPro, mais n'était pas optimisé pour le travail en laboratoire.

    \begin{flushleft}
        De ce fait nous avons conçu un dispositif permettant la capture vidéo avec divers fonctionnalités grace à l'application comme l'arrêt automatique des vidéos, prendre le nombre de photos souhaité en rafale, avec la possibilité d'avoir un rendu de la vidéo en temps réel dans les mains via à l'écran tactile.

        \vspace{0.2cm}

        Bien que nous perdons en qualité d'enregistrement vidéo et photo, nous avons réussit à réaliser un système optimal pour un moindre coût.

        \vspace{0.2cm}
    
        Mais le système n'est pas fixe, c'est à dire que nous pouvons changer le module camèra par un modèle plus performant afin d'avoir un meilleur rendu vidéo. 

        \vspace{0.2cm}

        Concernant l'application, nous pouvons améliorer la fait que l'application se bloque durant le mode rafale ou le mode d'enregistrement automatique en faisant du multi-processus.
        Cela nous permettrait de réaliser une tâche tout en faisant une autre.

        \vspace{0.2cm}

        Il serait également intéressant d'automatiser l'anotation d'image, en effet annoter une grande quantité d'image est une tache très longue.

        Avec la prise de vue en temps réele, nous pourions annoter anoter de manière automatique les images et en grande quantité sans que cela soit une tâche fastidieuse.
    \end{flushleft}


    
