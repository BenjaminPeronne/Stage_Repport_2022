\vspace*{\fill}
    \section*{Résumé}
        Dans ce rapport sont présentées les différentes étapes de conception du dispositif de capture vidéo et de la réalisation de l'application pour les biologistes du laboratoire ISYEB.
        Ce dispositif a pour but de remplacer celui existant du laboratoire ISYEB en offrant une meilleure polyvalence aux biologistes.
        L'application quant à elle doit permettre aux biologistes de réaliser différents types de prises de vue de manière plus efficace et autonome en fonction de leurs besoins.
        De ce fait, l'application doit permettre aux biologistes de lancer un enregistrement en ayant la possibilité de choisir l'emplacement et le nom du fichier vidéo enregistré. Permettre aussi la prise de photo en ayant la possibilité tout comme l'enregistrement vidéo de choisir l'emplacement et le nom du fichier photo enregistré.
        Ce dispositif doit permettre aux biologistes de récupérer les différents enregistrements via a différents types de périphériques externes de stockage en ayant également une possibilité de les visualiser depuis le dispositif.



    \section*{Abstract}
        In this report are presented the different stages of design of the video capture device and the realization of the application for the biologists of the ISYEB laboratory.
        The purpose of this device is to replace the existing one in the ISYEB laboratory by offering better versatility to biologists.
        As for the application, it should enable biologists to take different types of shots more efficiently and autonomously according to their needs.
        Therefore, the application must allow biologists to launch a recording by having the possibility of choosing the location and the name of the recorded video file. Also allow photo taking with the possibility, like video recording, of choosing the location and name of the saved photo file.
        This device must allow biologists to retrieve the different recordings via different types of external storage devices while also having the possibility of viewing them from the device.
\vspace*{\fill}