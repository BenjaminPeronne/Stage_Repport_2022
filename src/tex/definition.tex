\begin{definition}

    CSI : Camera Serial Interface (signifiant en anglais interface série pour caméra, CSI) est un standard d'interface électronique entre une caméra (un capteur ou une source vidéo) et un microprocesseur.
    
\end{definition}

\begin{definition}
    Overclocking : L'overclocking, ou parfois sur-cadencement1, ou surcadençage est une manipulation ayant pour but d'augmenter la fréquence du signal d'horloge d'un processeur au-delà de la fréquence nominale afin d'augmenter les performances de l'ordinateur.
\end{definition}

\begin{definition}
    Paradigme : Un paradigme de programmation est une façon d'approcher la programmation informatique et de traiter les solutions aux problèmes et leur formulation dans un langage de programmation approprié.
\end{definition}

\begin{definition}
    OS : Un système d'exploitation (ou système d'exploitation) est un ensemble de programmes et d'outils qui permettent d'exécuter des programmes sur un ordinateur.
\end{definition}

\begin{definition}
    API : L'API (Application Programming Interface) est une interface de programmation qui permet d'accéder à des fonctionnalités d'un logiciel.
\end{definition}

\begin{definition}
    GUI : La GUI (Graphical User Interface) est une interface utilisateur graphique qui permet d'interagir avec un logiciel.
\end{definition}